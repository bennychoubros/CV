\documentclass[12pt,a4paper]{moderncv}
\moderncvtheme[blue]{classic}
\usepackage[utf8]{inputenc}
\usepackage[inline]{enumitem}
% Package Geometry pour la gestion des marges
% Marge aux 4 coins de la page, ici elles sont réduites pour gagner de la place
\usepackage[top=1.0cm, bottom=1.0cm, left=1.6cm, right=1.6cm]{geometry}
% Largeur de la colonne de gauche pour les dates
\setlength{\hintscolumnwidth}{2.7cm}
\firstname{Benoît}
\familyname{Viala}
\title{Développeur / Devops}
\address{6 rue Georges Huchon}{94300 Vincennes}
\email{benoit@viala.tech}
% Demo / Portfolio :
% \homepage{www.viala.tech}
% \homepage{www.bennychou.fr}
% Social Media :
% Linkedin
% \social[linkedin]{ben.viala}
\social[github]{bennychoubros}
\mobile{06 63 24 86 67}
\extrainfo{28 ans -- Permis B}
\photo[80pt][0.4pt]{img/Portait.png}
% Designed to be a quote : I use it as a sub-title instead
\quote{Développeur Open-source enthousiaste, motivé par des philosophies Agiles et
DevOps.}
\begin{document}
\maketitle{}
% Negative margin to earn place between Title and Experience if needed
% \vspace*{-2.5\baselineskip}
\section{Expériences}
\section{Formations}
% Designed Template from the package :
% \cventry{year--year}{Degree}{Institution}{City}{\textit{Grade}}{Description}  % arguments 3 to 6 can be left empty
% My Template :
% \cventry{year--year}{Institution}{City}{Country (if needed)}{}{Description}  % arguments 3 to 6 can be left empty
\cventry{2016 à Aujourd'hui}{Ecole 42}{Paris}{}{}{%
\begin{itemize}%
\item "Apprendre à coder en réecrivant UNIX"
\item C, C++, Bash, Python / Django, POSIX
\item Apache, PostgreSQL / MySql
\item Docker / Virtualbox
\item Notion réseau / Firewall / Sécurité
\end{itemize}}
\cventry{2009 - 2015}{E.I.Purpan}{Toulouse}{}{}{Spécialité en Management de la Distribution des Produits alimentaire}
\cventry{2013}{Purdue University}{Indiana}{USA}{}{Semestre d’études en Collège d’Agriculture}
\section{Certifications}
\cventry{2017}{Digital Entrepreneurship Certificate}{HEC}{}{}{Master's Program créé par
\textit{{\href{https://www.hec.edu/en/master-s-programs/certificates/digital-entrepreneurship-certificate}{HEC
Paris}.}} 2 mois pour développer un projet de start-up. Formation axée sur l'entrepeneuriat, UX/UI,
Web design et Marketing.}
\section{Compétences}
\cvitem{Programmation}{Connaissances solides en C, Shell Script (Bash), Python.\newline{}%
Opérationnel en C++, Python / Django, Shell Script (Bash), HTML, CSS.}
\cvitem{OS}{Linux, Debian, MacOS X}
\cvitem{Cadre Travail}{Agile (SCRUM), cycle en V, TDD, TU}
\cvitem{{Admin services}}{Apache, Nginx, PostGreSQL,  MySQL.}
\cvitem{Virtualisation}{Docker, Virtualbox} 
\cvitem{CI}{Gitlab, Jenkins} 
\cvitem{Outils}{SVN/GIT, vim, make, dtrace, gdb, Valgrind, LaTeX}
\cvitem{Documentation}{Mantis, Redmine, Sphinx, Doxygen, DO-178B}
\cvitem{Sécurité}{Wireguard, SSH protocol}
\cvitemwithcomment{Anglais}{Bilingue}{TOEIC : 965 \texttt{/} 990 - 18mois en pays anglophones}
\section{Centres d'intérêt}
\cvitem{}{Rugby, Peinture, Electronique (Raspberry Pi...), Œnologie}
\end{document}
